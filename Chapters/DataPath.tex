\chapter{Datapath}
\begin{definition}[Datapath]
    A functional unit used to operate on or hold data within a processor. \textbf{Contorl unit} is an FSM that schedules data movement in datapath.
\end{definition}

\begin{example}
    Some datapath elements in MIPS
    \begin{enumerate}
        \item Instruction memory
        \item Data memory
        \item Register file
        \item ALU
        \item Adders
    \end{enumerate}
\end{example}

PC gets updated automatically which needs another adder so that it always points to the next instruction.
\section{Datapath elements}
\subsetction{Instruction fetch unit}
Contains PC which supplies the address for the memory. plus the adder plus the jump adder

\subsection{Instruction memory}
Cause it is slow, we put a cache. takes address from PC and returns an instruction. puts it on IR (note that for a single-cycle it is not a register )
decoding happens at the end.

\subsection{Register file}
Also called general purpose registers. Takes 3 indices, two for reading and one for writting, and data to write which is controlled by a signal. returns two data from the reading of the reading indices.

\subsection{ALU}
takes two input from RF (or IR in case of immediate) and returns the result of the operation which is determined be 4-bit select.

\subsection{Data memory}
Also known as D-Cache. two signals for reading and writting. one address input and data writting. returns data read if enabled.

D-cache and I-cache can be :
\begin{enumerate}
    \item completely separate components
    \item seperate components but share bus
    \item shared component and bus
\end{enumerate}
the first case it is easy.

\subsection{Branch unit}
for branching which includes sign extend unit, a shifter, and an adder.